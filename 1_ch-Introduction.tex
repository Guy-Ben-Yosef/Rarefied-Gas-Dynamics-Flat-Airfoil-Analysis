\chapter{Introduction}
The investigation of the hydrodynamic patterns surrounding a flat airfoil in unsteady flow represents a classic challenge in fluid mechanics that has been thoroughly examined as a prototypical case for understanding the dynamics of fluid flow over thin airfoils or fins\cite{gombosi1994gaskinetic, abramov2020rarefied, liu2021aerodynamic}. The study of fluid dynamics has always been an essential aspect of scientific research, as it plays a significant role in understanding the behavior of fluids around solid bodies. In recent years, some of the focus of fluid dynamics has shifted back towards investigating rarefied gas dynamics\cite{sazhin2023rarefied,jha2023heat}, which occurs at low pressures, high altitudes or very small-scale problems. In this context, the present study aims to investigate the unsteady rarefied gas dynamics over a flat airfoil using analytical techniques.

In the research endeavor entitled "Analytic Investigation of Unsteady Rarefied Gas Dynamics over a Flat Airfoil: Hydrodynamic Field Analysis," the objective is to scrutinize the hydrodynamic fields manifesting around a flat airfoil within the context of unsteady rarefied gas flow conditions. Furthermore, a comprehensive assessment of the lift-to-drag ratio will be conducted to elucidate the aerodynamic performance of the airfoil under the aforementioned conditions. Additionally, this study encompasses an examination of the far field, a critical component for the design and optimization of aerodynamic configurations.

The significance of this research project is multifaceted, as it adds to the current understanding of rarefied gas dynamics, a vital area of study for spacecraft and high-altitude aircraft design. Moreover, this study's findings have the potential to enhance the efficiency and performance of existing aerodynamic structures. Through a comprehensive analysis of the hydrodynamic fields around a flat airfoil, this research project addresses a wide range of velocities, spanning from subsonic to supersonic and ultrasonic regimes.

Furthermore, this research project aligns with ongoing developments in the field of micro-electro-mechanical systems (MEMS), which have led to an increasing number of investigations\cite{kirby2010micro, berthier2010microfluidics, chakraborty2012microfluidics, karniadakis2006microflows} on microfluidic flows encountered in small-scale devices with micro-channel geometries. The overwhelming complexity of channel networks contained in microfluidic chips has motivated researchers to explore the effects of various factors on rarefied gas flows. These factors include different shapes geometrical irregularities.

In the context of rarefied gas dynamics, these studies have been crucial in understanding the behavior of gas flows in micro-scale environments, and have shed light on the unique challenges and phenomena associated with rarefied gas flows in such geometries. This research project builds upon this body of knowledge by providing further insights into the hydrodynamic fields around a flat airfoil under unsteady rarefied gas flow conditions, expanding the understanding of rarefied gas dynamics in micro-scale environments and its potential applications in microfluidic devices and other related fields.

A substantial body of research has previously explored the two-dimensional problem of rarefied gas dynamics\cite{manela2021propagation, aoki2001rarefied}. Additionally, various investigations have approached this problem from a numerical standpoint, employing methods such as direct simulation Monte Carlo (DSMC)\cite{shoja2014investigation, fan2001computation, aoki1997numerical}. Building upon these works, this research project offers insights into the behavior of hydrodynamic fields under unsteady rarefied gas flow conditions, thereby expanding the current knowledge base in this area of study. In addition, this research project extends the scope of investigation to encompass a wide range of velocities, including subsonic, transonic, supersonic, and ultrasonic regimes. Moreover, the analysis goes beyond velocity considerations and includes an exploration of various angles of attack and temperature fluctuations of the airfoil, providing a comprehensive and thorough investigation of the hydrodynamic fields around a flat airfoil under unsteady rarefied gas flow conditions.

The results obtained from this research project should provide a better understanding of the behavior of rarefied gas flows over flat airfoils and can be used to optimize the design of aircraft and space vehicles operating under these conditions. Furthermore, the findings of this study will contribute to the development of computational models and simulation techniques for the analysis of rarefied gas flows, which can aid in the design and optimization of a wide range of engineering applications.

In summary, this research project is focused on investigating the hydrodynamic fields developed around a flat airfoil under the regime of unsteady rarefied gas flow. The study includes an analysis of the lift over drag ratio and the far field, which are crucial for the design and optimization of aerodynamic structures. The results of this study are expected to contribute to the field of fluid dynamics and have practical applications in the design of high-altitude aircraft and spacecraft.