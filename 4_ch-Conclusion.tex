\chapter{Conclusion}
\section{Conclusion}

The comprehensive examination of hydrodynamic patterns around a flat airfoil in unsteady flow presented in this study enriches our understanding of fluid mechanics. One of the primary endeavors was to elucidate the unsteady rarefied gas dynamics over a flat airfoil using intricate analytical techniques, and this investigation has notably enhanced our insights into the hydrodynamic fields. Given the pivotal nature of such investigations for spacecraft and high-altitude aircraft design, the significance of this research is manifold.

By developing an exact solution for the hydrodynamic fields within the ballistic limit, we have managed to address a broad spectrum of velocities, spanning from subsonic to ultrasonic regimes. This study's findings further underscored the intriguing behavior of hydrodynamic fields around an airfoil, especially when subjected to temperature variations or vertical velocities, culminating in the observation of dipole-induced dynamics (Fig. \ref{fig:dipole}). Additionally, the profound analysis of lift and drag dynamics as a function of angle of attack offers a promising platform for future aerodynamic explorations and optimizations.

From a methodological vantage point, the two-dimensional configuration paired with the normalization of magnitudes facilitated a comprehensive portrayal of the gas state. The dimensionless Boltzmann equation, particularly in the high Knudsen number regime, emerged as an instrumental tool in understanding the behavior of rarefied gases surrounding the airfoil, as indicated by Eqs. \ref{eq:Boltzman}, \ref{eq:macro_number_density}, \ref{eq:macro_velocity} and \ref{eq:macro_stress}. Notably, to determine the airfoil functions $\rho_\af^\pm(t,x)$ as delineated in Eq. \ref{eq:phi}, we imposed an impermeability condition on the normal velocity component $v(t,x,y)$.

Among the paramount findings, our observations during symmetric and steady-state flows highlighted a unique phenomenon, often referred to as the "density wedge." Evident in Fig. \ref{fig:steady_state} and Fig. \ref{fig:steady_state_superSonic}, this wedge characterizes a distinct variation in the density hydrodynamic field, resembling a wedge or a "V" shape behind the trailing edge of the airfoil. Our rigorous efforts enabled us to approximate this behavior with remarkable accuracy, especially for large $y$ values as showcased in Fig. \ref{fig:wedge_analysis}. 
Furthermore, our exploration into the density and static pressure fields on the line $x=1/2$ — abeam the chord center — allowed for the development of accurate approximations of these fields for large $y$ values, as depicted in Fig. \ref{fig:ACC_comparison}. 
Such insights hold profound implications, especially in the context of foreseeing far-field density and static pressure for specific challenges. By potentially neutralizing dynamic effects acting upon the airfoil.

While this research provides a robust foundation, the absence of a comparative analysis with other conditions or geometries signifies an area ripe for future exploration. There exists a pressing need to juxtapose these analytical insights with numerical approaches, such as DSMC, and the potential revelations from continuum limits.

On the practical front, the implications of this research extend far into the domain of aerospace engineering. The ability to theoretically predict lift and drag for flat airfoils in ballistic conditions stands as a testament to the study's relevance. Furthermore, the insights gained into neutralizing dynamic effects present promising prospects for reducing both power consumption and noise, especially in applications such as MEMS sensors.

However, the journey was not devoid of challenges. Navigating through a myriad of parameter permutations presented its set of complexities, suggesting potential avenues that remain unexplored.

In conclusion, this research stands as a monumental stride in the realm of rarefied gas dynamics, presenting both profound insights and opening doors for further explorations in fluid mechanics.